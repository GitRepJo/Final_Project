\chapter*{Abstract}
This project takes place in the context of the research project \ac{ELLA} that is funded by the German Federal Ministry of Transport and Digital Infrastructure. In this project, automated  manoeuvring of inland navigation vessels in confined spaces such as flood gates and harbours is investigated with a simulated virtual environment and a subsequent verification of the simulation results with a model scaled small vessel of an actual inland navigation vessel. The aim of this thesis is to develop a software architecture that allows for the integration of virtual sensors in the navigation subsystem of a guidance, navigation and control system of a small vessel in the inland navigation sector. In addition to the integration of virtual sensors, the architecture should be easily extendable towards the utilization of real sensors.\\
% The environment contains a scaled height map of the project specific location and is modelled according to the geographic characterisations of the real environment
 
The system is realized within the framework supplied by ROS2 in conjunction with Node.js and a virtual environment that is set up in a game engine provided by Unity3d. As a first step a virtual environment is set up in Unity3d. A small vessel that is contained in the virtual environment is equipped with virtual sensors such as a \ac{LIDAR} and an \ac{IMU}. The sensors publish its data with customized messages via Node.js that serves as a transition, to ROS2. In ROS2 the distinction between the three subsystems of the \ac{GNC} system is made and the sensor data is preprocessed by the navigation subsystem if necessary.\\

The results show that the chosen frameworks and algorithms allow for virtual data generation and subsequent data processing and visualization. It is demonstrated that the designed software architecture allows for the modular integration of  both virtual elements and algorithms. Ways to extend the design are outlined as well as improvements related to the performance of the implemented modules.
\chapter*{Kurzfassung}

Diese Projektarbeit findet im Rahmen des Forschungsvorhaben \ac{ELLA} statt, das vom Bundesministerium für Verkehr und digitale Infrastruktur gefördert wird. In diesem Projekt wird das automatisierte Manövrieren von Binnenschiffen in engen Räumen wie Schleusen und Häfen mit einer simulierten virtuellen Umgebung und einer anschließenden Verifikation der Simulationsergebnisse mit einem maßstabsgetreuen Kleinfahrzeug eines realen Binnenschiffs untersucht. Ziel dieser Arbeit ist es, eine Softwarearchitektur zu entwickeln, welche die Integration virtueller Sensoren in das Navigationssubsystem eines Führungs-, Navigations- und Steuerungssystems eines Binnenschiffs ermöglicht. Neben der Integration virtueller Sensoren soll die Architektur auch für die Nutzung realer Sensoren erweiterbar sein. Realisiert wird das System innerhalb des von ROS2 bereitgestellten Frameworks in Verbindung mit Node.js und einer virtuellen Umgebung, die in einer von Unity3d Game-Engine umgesetzt wird. \\


In einem ersten Schritt wird eine virtuelle Umgebung in Unity3d erstellt. Ein Kleinfahrzeug, das sich in der virtuellen Umgebung befindet, ist mit virtuellen Sensoren wie einem \ac{LIDAR} und einer \ac{IMU} ausgestattet. Die Sensoren veröffentlichen ihre Daten mit Nachrichten über Node.js, das als Übergang dient, an ROS2. In ROS2 wird die Unterscheidung zwischen den drei Subsystemen des \ac{GNC}-Systems vorgenommen und die Sensordaten werden bei Bedarf durch das Navigationssubsystem vorverarbeitet.\\

Die Ergebnisse zeigen, dass die gewählten Frameworks und Algorithmen eine virtuelle Datengenerierung und anschließende Datenverarbeitung und Visualisierung ermöglichen. Es wird gezeigt, dass die entworfene Software-Architektur eine modulare Integration sowohl von virtuellen Elementen als auch von Algorithmen ermöglicht. Möglichkeiten zur Erweiterung des Designs werden ebenso aufgezeigt wie Verbesserungen in Bezug auf die Leistung der implementierten Module.

