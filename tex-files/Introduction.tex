
\chapter{Introduction}\label{Introduction}
  \section{Motivation}

With the increasing relevance of climate change new concepts for mobility are required. A change in mobility demands not only decreasing carbon emissions but also increasing flexibility in order to remain competitive in an international context. With the strategic plan "Transport 2050" the European Commission sets targets to increase mobility in the European Union and to reduce emissions at the same time. Among these targets is the requirement to shift freight transport from road to rail and water. This is to be implemented by 2050 for 30\% of freight transports over 300km \cite{Transport2050}. Within the European Union, Germany proposed a national master plan specifically for the future of inland navigation in which the importance of inland navigation for a sustainable future is further highlighted. One important aspect is to increase the transport of goods by inland navigation to reach 12,3\% of total transported goods until 2030 \cite{MasterplanBinnen}. These requirements can be set against a background of increasing congestion on the roads and at the same time overcapacity of transport by water. In the context of these changes, waterborne transport, and in particular inland navigation, must prove to be competitive both economically and in terms of sustainability \cite{Christa2020}.\\

The automatisation of inland navigation vessels up to the possibility of an unmanned operated or autonomous vessels can play a vital role to full fill this requirements. There are expected to be huge advantages in operating \acp{USV} in comparison to traditional operated vehicles. With the automatisation of the vehicles comes an reduction of crew members to control a vehicle and hence the operational cost are reduced. Furthermore the reliability and flexibility in harsh or otherwise sophisticated environments increases. Particularly in shallow waters this flexibility allows an adaption of the \ac{USV} towards the requirements of this environment \cite{Liu2016}.\\

The technological innovation efforts of the inland navigation sector comprised mainly the development of low-emission engines \cite{Christa2020}. However, research into the field of \acp{USV} has been pursued over the past two decades. Even so, most current research is conducted on relatively small vehicles with limited capabilities regarding autonomy, endurance, payload and power outputs and experimental research remains an important topic for the future \cite{Liu2016}. \\

Therefore, designing a vessel specifically for autonomous operation to evaluate technologies for the next generation of inland navigation vessels is an important step towards further innovations in this sector. More specifically, the vessel should be able to sense its environment and operational state to enable situational awareness and autonomous decision making, possibly with the application of machine learning. Simulated environments provide an opportunity to facilitate the process of sensor data generation by means of virtual sensors. Therefore a software architecture that supports both real world and virtual appliance is of great importance in the design process of such a vessel.  


 \section{Previous work}\label{previous}
  
 Within the research project \ac{LAESSI}, a driver assistance system was designed and implemented on an inland navigation vessel with the focus on a bridge collision warning system and instruments to visualize the internal and external state of the vessel. To localize the vessels position, a land based corrected \ac{GNSS} signal was used together with electronic charts and further information transmitted from a land based station as for example temporarily restrictions caused by construction sides and the current water level. Internal sensors comprise a inertial measurement unit, a height sensor and turn indicator. The sensors data together with the \ac{GNSS} data are processed and checked for integrity as errors in the positing data directly influences the rudder commands in the chosen system architecture \cite{LAESSI}. The sensors in this research project however do not allow the vessel a land station independent representation of its environment and thus the vessel depends on an established connection to a land based station to function properly.\\
 
 A first step towards a higher degree of automatisation with the focus on reducing crew members to operate a vessel was laid by the KU Leuven with the vessel "Cogge". A design for an automated vessel is proposed with focus on the three key technological design aspects industrial relevance, vessel-environment interactions and the motions control of the vessel. The design is in accordance with an EU co financed vessel design project Watertruck+ \cite{watertruck}, in which standardised and modular small barges and push boats allow a high degree of flexibility especially in small waterways. The sensors of the "Cogge" comprising \ac{GNSS}, \ac{IMU}, \ac{LIDAR} and stereo image sensors. The software for the motion control is organized in a multilayered software design. To determine the route, waypoints are generated based on an open source map which then can be utilized by the middle level control. Software for the middle level control is provided by the \ac{MOOS-IvP} system \cite{MOOS}, a set of open source software modules specifically for the operation of autonomous marine vehicles. To reduce the error between the current and the desired speed and direction, a \ac{PID} controller compares the \ac{GNSS} and \ac{IMU} sensor values with given data of the \ac{MOOS-IvP}. The \ac{PID} controller connected with a downstream \ac{PLC} is implemented in the lower level control. Experiments conducted with this platform however has tended to focus on following generated waypoints with optimizing course deviations\cite{Peeters2020}. The integration of sensors beyond \ac{IMU} and \ac{GNSS} and the ability to manoeuvre in confined spaces as necessary in weir and lock systems remains open. Furthermore, the outlined design is lacking a \ac{RADAR} system to generate spatial data for objects further away than the range of the \ac{LIDAR}. That also leaves open the question how to merge the \ac{RADAR} and \ac{LIDAR} data at its transition.\\
 
 With increasing interest in autonomous driving especially in the field of automotive during the recent years, the research into automotive systems that integrates a wide variety of sensors is considerably more detailed compared to the inland navigation sector or more generally compared to water bound vehicles. A software-stack framework that is called "Autoware" provides an example of such a system designed to enable autonomous driving of vehicles. It is based on \ac{ROS}, an opensource middleware framework developed for robot applications and is open source itself. The required sensors that are used by the software can be purchased on the market and comprise a \ac{LIDAR}, a camera, \ac{GNSS}  with \ac{RTK} in the proposed system. On the software side, the tasks of algorithms are divided into the classes scene recognition, path planning and vehicle control. For each of these classes a set of algorithms is proposed. The software is run on an Nvidia \ac{GPU}, which allows for example for a faster object detection compared to \ac{CPU} based computation \cite{Autoware}. In a subsequent paper the migration of the software stack to an embedded system on board of a vehicle is proved and insofar its hardware flexibility is highlighted \cite{AutowareEmbedd}. The integration into an inland navigation vessel with its requirements and external influences remains open.
 
 \section{Outline}\label{Ziel}
 
 The present document is organized in five chapters. The first chapter (\ref{Introduction}) gives an introduction to the context of this thesis as well as a brief overview of previous research within the field of software architectures for inland navigation vessels.\\
 
 The second chapter (\ref{Fundamentals}) explains \ac{LIDAR} and \ac{RADAR} systems as well as the algorithms that are used to process its data. For the \ac{LIDAR}, direct, cluster based, hierarchic and density based methods are elaborated on that can be applied to cluster pointclouds and derive a structured representation of the environment in the around the sensor. To process the data of the \ac{RADAR}, algorithms for a fixed threshold and dynamic threshold signal filtering are presented. Three different implementations of the dynamic threshold are further outlined.\\
  
 Oriented on the V-Model for the development of safety related software, chapter three (\ref{Impl}) elaborates on the design of the software architecture for virtual sensors. First the overall system is designed with respect to appliances beyond this thesis. This is followed by a detailed description of the part that is implemented inside the system with a focus on the four sensor types that are integrated into the system. \\ 
 
 The system as designed in chapter three (\ref{Impl}) is tested and discussed in chapter four (\ref{Result}). Module and integration tests are conducted with emphasis on the performance of the system with the latency as an criteria. With a subsequent software validation, the implemented system is tested against the requirements.\\
 
 With a summary and outlook given in chapter five (\ref{ConFut}), this thesis is concluded.
 