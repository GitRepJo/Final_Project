
\chapter{Conclusion and Outlook}\label{ConFut}
 	\section{Conclusion} \label{Con}
    
    In this thesis a software architecture for virtual sensors in autonomous small vessels is conceptualized and implemented that allows for the integration of virtual and physical sensors.  A structured and safe design process is obtained by following the V-model for software designs. To avoid possible restrictions imposed by the limitations of the chosen software frameworks, a focus is laid on the system specification. With \ac{ROS2}, Unity3d and Node.js a solution is specified which not only meets the requirements but opens up the opportunity for further implementations that expand the scope of the introduced system. In conjunction with the system design, an Unity3d scene is created that resembles the topographic as well as visual data from an existing location. Furthermore, a viable way is shown to reconstruct and implement a \ac{RADAR} in a virtual environment. By integrating four different virtual sensors into the scene, the scalability of the design is demonstrated. It is shown, that algorithms in \ac{ROS2} which comprise a cluster analysis (\ac{DBSCAN}) and a false alarm detector (\ac{SOCA-CFAR}) are able to process the environmental data of the sensors and thus functions are implemented that are vital for the navigation subsystem of a \ac{GNC} system. Furthermore, routines that calculate bounding boxes around clustered \ac{LIDAR} data provides condensed environmental data and a way is shown to connect to the guidance subsystem of a \ac{GNC} system with a low data rate.\\
    
    In compliance with the defined requirements, the final result of this thesis is an architecture that allows for the easy incorporation of software that is vital to the design of autonomous inland navigation vessels. In addition, a variety of tested algorithms demonstrate the diverse opportunities that result of the chosen solution.
    
 	\section{Suggestions for Future Work} \label{Sug}
	Further waypoints of detailing the proposed architecture are to be found in the specified functions of a \ac{GNC} system. Starting with the implementation of a guidance subsystem that utilizes the information of the guidance subsystem, also a control subsystem has to be realized that connects to the actors of a vessel. It remains open which algorithms prove to full fill the required functionality of this subsystems. In this context, modelling the virtual scene in greater detail is clearly beneficial to the quality of information that can be derived from the simulation and exported to the behaviour of a real vessel. Especially adding a more sophisticated simulation of water is expected to increase the possibilities of testing algorithms closer to conditions that may occur in a real world scenario.  With the architecture providing an easy way to embed real sensors, it remains open how the system reacts to a substitution of the virtual sensors and influences of a real environment.\\
	
	 With the focus on demonstrating the versatility of the architecture, the focus in the implementation is on a variety of use cases instead of optimizations. Therefore, concerning the realized software, a number of possible improvements show up e.g. improving the GPU utilization in Unity3d. The interested reader may refer to appendix \ref{Attachlistb} where possible improvements to the existing software are detailed.
